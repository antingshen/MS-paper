
Intro \& our method of user studies (not as good as VATIC though)

The productivity of annotators is the focus of most of the prior works in this area.
During the development and deployment of BeaverDam, we've gathered extensive feedback from our turkers and testers.
We also implemented features suggested by past works such as VATIC,
and introduce several new ones to address issues found during the testing of VATIC.

\section{Keyframe scheduling \& Multiple object annotation}

Keyframe viewer when on custom schedule

\section{Video playback for maintaining identity}

VATIC's video player was a rudimentary image changer as its purpose was to give workers context when tracking objects.
As we wanted more flexibility and power for more advanced labelers, we incorporated a full fledged video player instead of using VATIC's system of using JavaScript to advance images of frames.

One huge issue we solved with this setup was video loading times.
When deploying VATIC, we found users complaining about load times, and sometimes jobs would time out before the video loaded, especially when using high resolution images for frames.
When seeking in the video, the images would flash to white if it's not loaded from cache quickly enough.
By converting the app to use HTML5 video, we take advantage of its ability to stream, similar to YouTube and otger sites can display the video without loading all of it.
The annotator can then begin annotating as the video loads.
The playback is also much smoother, with a minor tradeoff of slightly slower rewinds, as videos are encoded in a way optimized for forward playback.

\section{Reducing clicks}

In addition to measuring time it takes for annotators to perform certain actions, we sought to minimize the number of clicks necessary.
We find this to be a good rule of thumb when we dont have enough users to properly A/B test design choices.

First, we set the user in create/edit hybrid mode by default.
A new object is created if the user drags in an empty spot, and an existing object is edited otherwise.
VATIC requires the user to click a ``create'' button, which enables the user to annotate heavily overlapped objects more easily.
We propose that it's more efficient to force a user to perform the extra steps of creating a box in an empty spot and dragging it onto the object in case of overlap,
as heavy overlaps turned out to be rare in our datasets, so it was better to optimize for the common case.

Additionally, instead of having the user choose the type of object each time, we default to the most common label (car) or the user's last selection.
This reduced clicks, but did result in more errors for new users who didn't notice that they must change the label from the default when labeling non-cars.
While VATIC's prompting each time prevented this, we elected to solve it through a better tutorial to save time for trained annotators.

Finally, BeaverDam includes extensive keyboard shortcuts, with the aim of eliminating any need for the mouse aside from drawing and moving boxes. 
Tasks such as label selections and video playback are all controlled through the keyboard.
We left in optional mouse controls, but it may be a good choice to remove them to enforce keyboard shortcut usage.
Surprisingly, we have had turkers report that they were on tablets, and keyboard shortcuts were inefficient for them.
However, we find in our small sample of annotators that tablet users are few and are less efficient in general.

\section{Handling frame exit/enters}

During our user tests, we found that users have the most difficulty when objects are entering or exiting the frame, as it's difficult for linear interpolation to handle a box whose size is increasing but one edge is fixed to the edge of the frame. 
This was one of the main challenges when annotating using VATIC, as we had many driving videos where objects frequently entered and exited.
To address this issue, we introduced two solutions.
First, we allow boxes to be partially or wholly located outside the frame. 
This enables the user to guess at the true location of the object if the frame were bigger, which allows for much better linear interpolation <INSERT EXAMPLE>
Then, we add a large padding around the border of the frame, essentially enlarging the frame with blank space, which makes drawing and editing boxes that are located partially outside the frame much easier. 
After labeling, any boxes that are partially outside the frame can be cropped to the edge of the frame.

\section{Micro vs Macro tasks}

Comparison from existing literature

Proposal advocating for micro-tasks in video labeling

Extensible task structure

\section{Interpolation \& Tracking}
