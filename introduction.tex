\section{Spectrum allocation vs.~usage}
\label{sec:usage}

Looking at the NTIA's chart of these frequency allocations
(Figure~\ref{fig:spectrum_alloc}a), it appears that we are in danger
of running out of spectrum~\cite{NTIAspectrum}. However, allocation
is only half the story. Contrary to popular belief, actual
measurements (taken in downtown Berkeley, CA) show that most of the
allocated spectrum is vastly underutilized
(Figure~\ref{fig:spectrum_alloc}b)~\cite{BWRC2004, NewAm2005}.

\begin{figure}[htp]
\begin{center}
\subfigure[What does this graph mean?]{
  \epsfxsize=3in \epsfbox{spectrum2.eps}}
\subfigure[This picture makes no sense here.]{
  \epsfxsize=2.8in \epsfbox{usage.eps}}
\caption{\footnotesize{This picture is just here as a placeholder.
\label{fig:spectrum_alloc}}}
\end{center}
\end{figure}

blah blah blah...

Examining solely the 402 MHz allocated to broadcast TV
(Appendix~\ref{appen:commspec})...

\section{The policy debate}
\label{sec:policy}

Clearly, the spectrum is far from fully utilized. As a result, the
FCC's exclusive-use allocation policy is being increasingly viewed
as outdated.
