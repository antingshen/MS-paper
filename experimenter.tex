
During our investigation, we had many frustrating experiences with existing research tools in this area.
These included installation issues, configuration issues, and dealing with unintuitive and undocumented interfaces.
For example, there are dozens of open issues without solutions on the VATIC GitHub, 
and its installation script fails at multiple points on a brand new Ubuntu box. 
Due to preciousness of researcher's time,
we believe the ability for researchers to test and iterate quickly is just as important as worker speeds when it comes to video annotation.
Therefore we have placed improving the experimenter's experience as one of the main goals of this work.

\section{Interface for researcher}

For a basic user creating a crowdsourced video dataset using the default configurations we used, 
BeaverDam provides a streamlined interface.
We provide a setup script that is tested on clean installs of Ubuntu 14.04 and Ubuntu 16.04.
In comparison, VATIC was tested on an unspecified Ubuntu installation with exact Apache and MySQL installs,
and while the install script claims that it should work on any operating system in theory, 
installation is difficult in reality. 
Additionally, our install script configures everything from Nginx and TLS to database config and backups.
The user only needs to place their keys and credentials in the locations specified in our documentation.
While a containerization system such as Docker would have also solved VATIC's issues and ensure future compatibility,
we felt that the additional complexity is not worth it, as many of our users in the research community are unfamiliar with Docker. 

To use BeaverDam after installation, we provide a web interface for researchers to easily add and view videos and jobs.
We feel that this is superior to VATIC's command line based approach, 
as the number of flags needed to specify various configurations was overwhelming. 
However, to allow experimenters to load large number of videos or perform other tasks programatically, 
BeaverDam also provides a Python shell interface, backed by Django, to expose every functionality through Python.

Lastly, as BeaverDam is HTML5 video based, no frame extraction step is necessary.
H.264 encoded videos will work without preprocessing.
However, we do provide scripts to convert these videos to images with matching annotations to feed into machine learning tools if desired.

\section{Decoupled modules}

As the users of BeaverDam will most likely need to modify BeaverDam to fit their needs,
we've emphasized modularity in our designs. 
Since modularity is something VATIC did well with their turkic, vatic, and pyvision splits, 
we've taken a similar approach. 
Our infrastructure is split between a crowdsourcing module, an annotation module, and a CV-based tracking module. 
But we decided to go a step further and make other parts of our platform replaceable as well.

To serve videos, we've provided Nginx to allow VATIC users to continue serving videos efficiently on the same server as their application.
But to allow for scalability, users can use AWS S3 or any other CDN, or even a mix, by specifying so when creating jobs.
Similarly, our setup pipeline is done through Ansible, a configuration management tool popular in industry. 
This allows users to deploy on their own servers, or in the cloud.

We also understand that not everyone wants to use Amazon Mechanical Turk. 
While past research has proven Mturk to be efficient and reliable, 
companies in industry seeking training data may prefer alternatives such as CrowdFlower, 
or may even choose to label in-house, or outsource to contractors.
BeaverDam is designed with this flexibility in mind as it carries its own authentication and works independently of Mturk.
This allows companies with the resources to seek cheaper labor in other countries to use BeaverDam as a platform for their workers
and track progress and pay without the need for Amazon Mechanical Turk, which carries a 20\% fee. 

\section{Patterns}

Django backend

Patterned frontend (MVC, event-based)

\section{Dependencies}

Few dependencies (sqlite)

Easily restore state

Uses newer technologies

\section{Security}

DB Backup

Authentication enables decoupling, web admin

Other standard procedures, HTTPS \& HSTS, CSRF, clickjacking


