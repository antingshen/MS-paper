We have presented numerous improvements to existing video annotation tools~\cite{Vatic}.
We first argued that optimizing for researcher time is as important, if not more important than optimizing for worker efficiency.
In this area, we identified several pain points of tools such as VATIC, including lack of modularization, extensibility, and ease of installation \& use.
We improved on each of these areas as well as others such as application security and speedy video distribution, all with scalable real world deployment in mind.
The way BeaverDam is engineered allows for new types of annotations and data, such as point clouds from LIDAR, to be labeled in the future with few changes.
Aside from these software product \& implementation improvements, we demonstrated several ideas discovered in our experiments that allow for easier annotation by the workers.
We improve the keyframe scheduling interface, drastically reduced video loading times (especially for workers with slow connections), reduced clicking required by 20-40\%, reduced time necessary to label videos using smarter defaults and keyboard shortcuts, and improved the handling for occluded objects and objects that are partially in frame.
These improvements in BeaverDam enables cheaper annotations to create larger datasets to improve model accuracies in production.
